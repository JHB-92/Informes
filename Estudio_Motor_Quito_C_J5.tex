\documentclass[12pt]{article}
\usepackage[utf8]{inputenc}
\usepackage[spanish]{babel}

%Fuente (compilarlo en Latex.pdf normal porque va más rápido y luego
%y al final insertarlo en Arial) DA ERROR
%\usepackage{fontspec}
%\setmainfont{Arial}

%Estructura de la Página
\usepackage[left=2.54cm,right=2.54cm,top=2.54cm,bottom=2.54cm]{geometry}

%Tablas

\usepackage[]{float}

%Páginas en horizontal


%Pies de página y encabezado
\usepackage{fancyhdr}

%Comentario párrafos
\usepackage{verbatim}

%Paquete arquitectura página
\pagestyle{fancy}
\fancyhf{}
\lhead{Revisión Informe Ensayos Lote 2}
\rhead{CAF}
\rfoot{\thepage}
\lfoot{}

%Interlineado
\renewcommand{\baselinestretch}{0.5}



%Paquete Matemático
\usepackage{amsmath}
\usepackage{amsfonts}
\usepackage{amssymb}
\usepackage{breqn}

%Paquete para el código
% Paquetes Listing

\usepackage{listings}
\usepackage{xcolor}

%Settings 

\definecolor{codegreen}{rgb}{0,0.6,0}
\definecolor{codegray}{rgb}{0.5,0.5,0.5}
\definecolor{codepurple}{rgb}{0.58,0,0.82}
\definecolor{backcolour}{rgb}{0.95,0.95,0.92}


\lstdefinestyle{mystyle}{
 backgroundcolor=\color{backcolour},   
 commentstyle=\color{codegreen},
 keywordstyle=\color{magenta},
 numberstyle=\tiny\color{codegray},
 stringstyle=\color{codepurple},
 basicstyle=\ttfamily\footnotesize,
 breakatwhitespace=false,         
 breaklines=true,                 
 captionpos=b,                    
 keepspaces=true,                 
 numbers=left,                    
 numbersep=5pt,                  
 showspaces=false,                
 showstringspaces=false,
 showtabs=false,                  
 tabsize=2
}

\lstset{style=mystyle}
%Paquete Imágenes

\usepackage{graphicx}
\graphicspath{ {Images/} }
\usepackage{subfig}

%Bibliografía
\usepackage[backend=bibtex]{biblatex}
\addbibresource{referencias.bib}

%Espacio entre párrafos
\setlength{\parskip}{0.5cm}
%Sangría
\setlength{\parindent}{0cm}

\title{Estudio de los rendimentos declarados  obtenidos para el motor de TSA implementado en el proyecto C.J5}
\author{José Honrubia Blanco}
\date{Noviembre 2022}

\begin{document}
\section{Objetivo}
El objetivo de este documento es comparar los valores calculados del motor TSA usado en el proyecto C.J5 con los valores finalmente obtenidos

\section{Método de Cálculo}
Los cálculos se van a realizar para las frecuencias de 30, 62.2, 71, 75 y 95 Hz. Al final, se hará la comparativa entre los puntos recogidos en el datasheet y los obtenidos en los ensayos realizados por TSA.

Para la ilustración de los cálculos se va a emplear los datos de partida de 75 Hz por ser el punto característico del motor (S1). Los valores se recogen en la tabla \ref{table: valores_iniciales}:
{\extrarowheight
\renewcommand{\arraystretch}{2.25}
    \begin{table}[H]
    \centering
    \begin{tabular}{ccccc}
    P2 (W) & \eta  (\%) & s (\%) & $M_{calculado}$ (Nm) & $M_{medido}$ (Nm) \\ \hline
    1026   & 15.7    & 0.01   & 4.4             & 4            \\
    2054   & 27.2    & 0.02   & 8.7             & 8            \\
    2946   & 34.9    & 0.03   & 12.5            & 5.5          \\
    4480   & 44.8    & 0.04   & 19              & 13           \\
    7373   & 57.1    & 0.05   & 31.3            & 22.75        \\
    9778   & 63.8    & 0.05   & 41.5            & 32           \\
    16875  & 75      & 0.09   & 71.7            & 58.5         \\
    23811  & 80.7    & 0.12   & 101.1           & 96.5         \\
    30405  & 84      & 0.16   & 129.2           & 125.5        \\
    42870  & 87.8    & 0.22   & 182.2           & 176          \\
    68520  & 91.4    & 0.33   & 291.5           & 277.5        \\
    98666  & 93.2    & 0.48   & 420.4           & 421.25       \\
    119382 & 93.8    & 0.61   & 509.2           & 495          \\
    138462 & 94.1    & 0.72   & 591.1           & 594.5        \\
    158594 & 94.3    & 0.77   & 677.5           & 671.75       \\
    180824 & 94.4    & 0.93   & 773.6           & 762.75       \\
    201008 & 94.5    & 1.01   & 859.9           & 866.5        \\
    224159 & 94.3    & 1.21   & 962.7           & 958          \\
    250876 & 94.3    & 1.31   & 1078            & 1070.5       \\
    283611 & 94.1    & 1.45   & 1219.9          & 1202.5       \\ \hline
    \end{tabular}
    \caption{Valores aportados por TSA para la frecuencia de 75 Hz a distintas potencias}
    \label{tab: valores_iniciales}
    \end{table}}


\end{document}